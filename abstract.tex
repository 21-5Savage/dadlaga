\begin{abstract}
Энэхүү дадлагын тайлангийн явцад миний бие Бадарчийн Бат-Энх Unitel Group компанийн Дижитал  газрын Дата сайнсын хэлтэст дадлагын төлөвлөгөөнд хамаарах хугацаанд гүйцэтгэсэн судалгаа, хэрэгжилт болон тэдгээрийн үр дүнг танилцуулав. Дадлагын ажлын хүрээнд байгууллагын дотоод процессуудад хиймэл оюун ухаанд суурилсан шийдлүүдийг нэвтрүүлэх, шинэ архитектур болон техник ашиглан комплекс системүүдийн автоматжуулалт, өгөгдөлд суурилсан шийдвэр гаргалтыг оновчлох ажлууд хийгдсэн.

Судалгааны ажлын хүрээнд Agentic AI, Model Context Protocol, Huawei Cloud Stack, Langfuse, Gemini болон AWS Cohere зэрэг орчин үеийн технологи, архитектуруудын онолын үндэс болон практик хэрэглээг судалсан. Харин хэрэгжилтийн хэсэгт ERP системтэй холбогдож буй MCP сайжруулах, Information Retrieval MCP-д вектор өгөгдлийн санг шилжүүлэх, embedding моделиудыг сольж доголдлын үед ажиллагааг бүрэн хангах логик зэргийг хөгжүүлсэн аргачлалыг оруулав. Үүнээс гадна Huawei Cloud Stack дээр development болон production орчныг төлөвлөн байгуулах, CI/CD болон deployment infrastructure-ийг сайжруулах ажлуудыг хийж гүйцэтгэсэн. 

Гүйцэтгэсэн ажлуудын үр дүнд LLM-д суурилсан системүүдийн гүйцэтгэл, найдвартай байдал болон ажиглалт, хяналтын түвшин сайжирч, байгууллагын дотоод процессын автоматжуулалт, өгөгдлийн ашиглалтын үр ашиг нэмэгдсэн гэж дүгнэж байна.

Unitel компанийн дотоод бодлогын хүрээнд системийн нарийн зохион байгуулалт болон түүнд ашиглагдаж буй техникийн аргачлалын дэлгэрэнгүй тайлан болон дотоод судалгаа, туршилтын үр дүн болон тэдгээрт хамаарах өгөгдлийг тайланд оруулаагүй болно. 
\end{abstract}
