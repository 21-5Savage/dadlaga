%   Доорх хэсгийг өөрчлөх шаардлагагүй
%!TEX TS-program = lualatex
%!TEX encoding = UTF-8 Unicode
\documentclass[12pt,a4paper]{report}

\usepackage{fontspec}
\setmainfont[Ligatures=TeX]{Times New Roman}
\setsansfont{Arial}

% \usepackage[utf8x]{inputenc}
% \usepackage[mongolian]{babel}
%\usepackage{natbib}
\usepackage{geometry}
%\usepackage{fancyheadings} fancyheadings is obsolete: replaced by fancyhdr. JL
\usepackage{fancyhdr}
\usepackage{float}
\usepackage{afterpage}
\usepackage{graphicx}
\usepackage{amsmath,amssymb,amsbsy}
\usepackage{dcolumn,array}
\usepackage{tocloft}
\usepackage{dics}
\usepackage{nomencl}
\usepackage{upgreek}
\newcommand{\argmin}{\arg\!\min}
\usepackage{mathtools}
\usepackage[hidelinks]{hyperref}

\usepackage{algorithm}
\usepackage{algpseudocode}

\usepackage{listings}
\DeclarePairedDelimiter\abs{\lvert}{\rvert}%
\makeatletter
\usepackage{caption}
\captionsetup[table]{belowskip=0.5pt}
\usepackage{subfiles}

\usepackage{listings}
\renewcommand{\lstlistingname}{Код}
\renewcommand{\lstlistlistingname}{\lstlistingname ын жагсаалт}

\usepackage{xcolor}
\definecolor{codegreen}{rgb}{0,0.6,0}
\definecolor{codegray}{rgb}{0.5,0.5,0.5}
\definecolor{codepurple}{rgb}{0.58,0,0.82}
\definecolor{backcolour}{rgb}{0.99,0.99,0.99}
 
\lstdefinestyle{mystyle}{
    basicstyle=\ttfamily\small,
    backgroundcolor=\color{backcolour},   
    commentstyle=\color{codegreen},
    keywordstyle=\color{magenta},
    numberstyle=\tiny\color{codegray},
    stringstyle=\color{codepurple},
    %basicstyle=\footnotesize,
    breakatwhitespace=false,         
    breaklines=true,                 
    captionpos=b,                    
    keepspaces=false,                 
    numbers=left,                    
    numbersep=10pt,                  
    showspaces=false,                
    showstringspaces=true,
    showtabs=false,                  
    tabsize=2
}

\lstset{style=mystyle}

\lstdefinestyle{mpython}{
    language=Python,
    basicstyle=\ttfamily\small,
    keywordstyle=\color{blue},
    stringstyle=\color{red},
    commentstyle=\color{gray},
    numberstyle=\tiny\color{gray},
    numbers=left,
    stepnumber=1,
    frame=single,
    breaklines=true,
    showstringspaces=false,
    tabsize=4
}

\let\oldabs\abs
\def\abs{\@ifstar{\oldabs}{\oldabs*}}
\makenomenclature
\begin{document}


%----------------------------------------------------------------------------------------
%   Өөрийн мэдээллээ оруулах хэсэг
%----------------------------------------------------------------------------------------

% Дипломийн ажлын сэдэв
\title{Үйлдвэрлэлийн дадлага}
% Дипломын ажлын англи нэр
\titleEng{Internship report}
% Өөрийн овог нэрийг бүтнээр нь бичнэ
\author{Бадарчийн Бат-Энх}
% Өөрийн овгийн эхний үсэг нэрээ бичнэ
\authorShort{Б.Бат-Энх}
% Удирдагчийн зэрэг цол овгийн эхний үсэг нэр
\supervisor{У. Нямбаяр}
% Хамтарсан удирдагчийн зэрэг цол овгийн эхний үсэг нэр
\cosupervisor{}

% СиСи дугаар 
\sisiId{22B1NUM7226}
% Их сургуулийн нэр
\university{МОНГОЛ УЛСЫН ИХ СУРГУУЛЬ}
% Бүрэлдэхүүн сургуулийн нэр
\faculty{МЭДЭЭЛЛИЙН ТЕХНОЛОГИ, ЭЛЕКТРОНИКИЙН СУРГУУЛЬ}
% Тэнхимийн нэр
\department{МЭДЭЭЛЭЛ, КОМПЬЮТЕРЫН УХААНЫ ТЭНХИМ}
% Зэргийн нэр
\degreeName{Бакалаврын судалгааны ажил}
% Суралцаж буй хөтөлбөрийн нэр
\programeName{Компьютерын Ухаан (D061301)}
% Хэвлэгдсэн газар
\cityName{Улаанбаатар}
% Хэвлэгдсэн огноо
\gradyear{2026 оны 2 дугаар сар}


%----------------------------------------------------------------------------------------
%   Доорх хэсгийг өөрчлөх шаардлагагүй
%----------------------------------------------------------------------------------------
%----------------------Нүүр хуудастай хамаатай зүйлс----------------------------
\pagenumbering{roman}
\makefrontpage
\maketitle
\doublespace

% Decleration
\begin{huge}
\textbf{Зохиогчийн баталгаа}
\end{huge} \\ \ \\ 
\doublespace
Миний бие \@author \ "\@title" \ сэдэвтэй судалгааны ажлыг гүйцэтгэсэн болохыг зарлаж дараах зүйлсийг баталж байна:
\begin{itemize}
\item Ажил нь бүхэлдээ Монгол Улсын Их Сургуульд дээд боловсролын зэрэг горилохоор дэвшүүлсэн болно.
\item Энэ ажлын аль нэг хэсгийг эсвэл бүхлээр нь ямар нэг их, дээд сургуулийн зэрэг горилохоор оруулж байгаагүй болно.
\item Бусдын хийсэн ажлаас хуулбарлаагүй, эшлэл, зүүлтийг зохистой хийсэн болно.
\item Ажлыг зохиогч би хийсэн ба миний хийсэн ажил, бусдын дэмжлэгийг дадлагын ажлын тайланд тодорхой тусгасан болно. 
\end{itemize} 
\ 

Гарын үсэг: \underline{\hspace{5cm}} 

Огноо: 	\ \ \underline{\hspace{3cm}}

% Гарчгийг автоматаар оруулна
\setcounter{tocdepth}{1}
\tableofcontents

% Зургийн жагсаалтыг автоматаар оруулна
\listoffigures

% Хүснэгтийн жагсаалтыг автоматаар оруулна
\listoftables

% Кодын жагсаалтыг автоматаар оруулна
\lstlistoflistings

% This puts the word "Page" right justified above everything else.
\newpage
%% \addtocontents{lof}{Зураг~\hfill Хуудас \par}
\newpage
%% \addtocontents{lot}{Хүснэгт~\hfill Хуудас \par}

\renewcommand{\cftlabel}{Зураг}


\doublespace
\pagenumbering{arabic}


% Удиртгалыг оруулж ирэх ба abstract.tex файлд удиртгалаа бичнэ
\begin{abstract}
Энэхүү дадлагын тайлангийн явцад миний бие Бадарчийн Бат-Энх Unitel Group компанийн Дижитал  газрын Дата сайнсын хэлтэст дадлагын төлөвлөгөөнд хамаарах хугацаанд гүйцэтгэсэн судалгаа, хэрэгжилт болон тэдгээрийн үр дүнг танилцуулав. Дадлагын ажлын хүрээнд байгууллагын дотоод процессуудад хиймэл оюун ухаанд суурилсан шийдлүүдийг нэвтрүүлэх, шинэ архитектур болон техник ашиглан комплекс системүүдийн автоматжуулалт, өгөгдөлд суурилсан шийдвэр гаргалтыг оновчлох ажлууд хийгдсэн.

Судалгааны ажлын хүрээнд Agentic AI, Model Context Protocol, Huawei Cloud Stack, Langfuse, Gemini болон AWS Cohere зэрэг орчин үеийн технологи, архитектуруудын онолын үндэс болон практик хэрэглээг судалсан. Харин хэрэгжилтийн хэсэгт ERP системтэй холбогдож буй MCP сайжруулах, Information Retrieval MCP-д вектор өгөгдлийн санг шилжүүлэх, embedding моделиудыг сольж доголдлын үед ажиллагааг бүрэн хангах логик зэргийг хөгжүүлсэн аргачлалыг оруулав. Үүнээс гадна Huawei Cloud Stack дээр development болон production орчныг төлөвлөн байгуулах, CI/CD болон deployment infrastructure-ийг сайжруулах ажлуудыг хийж гүйцэтгэсэн. 

Гүйцэтгэсэн ажлуудын үр дүнд LLM-д суурилсан системүүдийн гүйцэтгэл, найдвартай байдал болон ажиглалт, хяналтын түвшин сайжирч, байгууллагын дотоод процессын автоматжуулалт, өгөгдлийн ашиглалтын үр ашиг нэмэгдсэн гэж дүгнэж байна.

Unitel компанийн дотоод бодлогын хүрээнд системийн нарийн зохион байгуулалт болон түүнд ашиглагдаж буй техникийн аргачлалын дэлгэрэнгүй тайлан болон дотоод судалгаа, туршилтын үр дүн болон тэдгээрт хамаарах өгөгдлийг тайланд оруулаагүй болно. 
\end{abstract}


%----------------------------------------------------------------------------------------
%   Дипломын үндсэн хэсэг эндээс эхэлнэ
%----------------------------------------------------------------------------------------
%\addcontentsline{toc}{part}{БҮЛГҮҮД}
% Шинэ бүлэг
\chapter{Байгууллагын танилцуулга}
\subfile{subfiles/chapter1.tex}

\chapter{Судалгаа}
\subfile{subfiles/chapter2.tex}

\chapter{Хэрэгжүүлэлт}
\subfile{subfiles/chapter3.tex}


\chapter{Үр дүн}
\subfile{subfiles/result.tex}

%----------------------------------------------------------------------------------------
%   Дүгнэлт эндээс эхэлнэ
%----------------------------------------------------------------------------------------
\conclusion{Дүгнэлт}
\subfile{subfiles/conclusion.tex}

%----------------------------------------------------------------------------------------
%   Дипломын номзүй, хавсралтын хэсэг эндээс эхэлнэ
%----------------------------------------------------------------------------------------

%----------------------------------------------------------------------------------------
%   BIBLIOGRAPHY
%----------------------------------------------------------------------------------------
\bibliographystyle{plain}
\bibliography{references}

%----------------------------------------------------------------------------------------
%   Хавсралтууд эндээс эхэлнэ
%----------------------------------------------------------------------------------------
\appendix
\addcontentsline{toc}{part}{ХАВСРАЛТ}

% Хавсралтын нэр. Хавсралт гэдэг үг агуулахгүй
\chapter{}
Хавсралтын агуулга

% Хавсралтын нэр. Хавсралт гэдэг үг агуулахгүй
\chapter{}


\begin{lstlisting}[language=Python]
import numpy as np
 
def incmatrix(genl1,genl2):
    m = len(genl1)
    n = len(genl2)
    M = None #to become the incidence matrix
    VT = np.zeros((n*m,1), int)  #dummy variable
 
    #compute the bitwise xor matrix
    M1 = bitxormatrix(genl1)
    M2 = np.triu(bitxormatrix(genl2),1) 
 
    for i in range(m-1):
        for j in range(i+1, m):
            [r,c] = np.where(M2 == M1[i,j])
            for k in range(len(r)):
                VT[(i)*n + r[k]] = 1;
                VT[(i)*n + c[k]] = 1;
                VT[(j)*n + r[k]] = 1;
                VT[(j)*n + c[k]] = 1;
 
                if M is None:
                    M = np.copy(VT)
                else:
                    M = np.concatenate((M, VT), 1)
 
                VT = np.zeros((n*m,1), int)
 
    return M
\end{lstlisting}

\end{document}
