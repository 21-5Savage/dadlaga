Энэхүү дадлагын ажлын хүрээнд хиймэл оюун ухаанд суурилсан системүүдийг байгууллагын бодит хэрэгцээнд нийцүүлэн судалж, хэрэгжүүлэх практик туршлага олж авлаа. Agentic AI архитектур, Model Context Protocol, вектор өгөгдлийн сан, embedding модель болон cloud дэд бүтцийн зохион байгуулалт зэрэг орчин үеийн технологиуд дээр хийгдсэн онолын ба практикийн цогц ажлууд шинэ технологиудтай ажиллах маш чухал туршлага болсон.

Мөн production орчинд ашиглагдаж буй системүүдэд хяналт тавих, найдвартай байдлыг хангахын ач холбогдол, түүнчлэн дотоод дэд бүтцэд тохирсон технологийн сонголт хийх шаардлагыг практик түвшинд ойлгож чадсан. Үүнээс гадна server-ийн архитектур зохион байгуулах тэдгээрийн гадаад болон дотоод сүлжээний холболт зэргийн хянан ажилласан туршлага нь мэргэжлийн хичээлд төдийлөн тусгагддаггүй компьютер сүлжээний талаарх мэдлэгээ дээшлүүлэх боломж олголоо. 

Цаашид Agentic AI системүүдийг нарийвчлалтай ажлуудад ашиглах, хиймэл оюун ухаанд суурилсан автоматжуулалтыг бизнес процессуудад нэвтрүүлэх болон тэдгээрийн оршиж буй техник орчныг бүрдүүлэн хяналт тавих зэрэг дадлагын ажлын явцад олж авсан олон талын мэдлэгээ сайжруулан бусад судалгаа, хэрэгжүүлэлтэд дахин ашиглах бүрэн боломжтой гэж үзэж байна.