Unitel Group нь дэлхийн хиймэл оюуны ухааны хувьсгалтай ижил түвшинд хөгжих зорилгын дагуу шинэ технологи болон архитектуруудыг компаны хэрэгцээнд зориулан байнга нэвтрүүлэн ашигладаг.

\section{Agentic AI}
Agentic архитектур нь нэг ёсондоо өөрийгөө тодорхойлдог workflow юм \cite{ibm_agentic} \cite{google_agentic}. Large Language Model (LLM)-ийн огцом хөгжил нь энэ архитектурын динамик байдлыг хангаж комплекс ажлуудыг програмчлалын аргаар шийдэх боломжийг олгосон.

\begin{figure}[H]
\centering
\includegraphics[width=0.8\textwidth]{../images/workflow.png}
\caption{Workflow}
\end{figure}

Workflow нь статик програмчлагдсан бөгөөд зорилго болон зорилго биелүүлэхэд хэрэгцээтэй зүйлсийг тодорхойлох механизм байхгүй.
\begin{figure}[H]
\centering
\includegraphics[width=0.8\textwidth]{../images/agent.png}
\caption{Agentic}
\end{figure}

\section{Model Context Protocol}

Antropic компанийн хөгжүүлсэн тус протокол нь LLM-д tool call хийх боломж олгох хамгийн уян хатан технологи юм. Model Context Protocol ашигласнаар LLM нь гадны өгөгдөлтэй харьцах стандарт харилцааны протоколтой болсон. Үүний үр дүнд LLM-ийн модель болгон өөрийн гэсэн tool-тэй байхын оронд нэг tool-ийг олон модель зэрэг ашиглах, өөр хүмүүсийн бэлтгэсэн tool-ийг ашиглах зэрэг боломжууд үүссэн \cite{mcp}. 

\section{Huawei Cloud Stack}

Huawei Cloud Stack (HCS) нь Data Center удирдлагын систем бөгөөд Unitel компани өөрийн Data Center-ээ тус системээр удирддаг бөгөөд нэвтрүүлэлтийн болон хөгжүүлэлт ажлуудыг удирдахад тус системийн тухай мэдлэгтэй байх шаардлагатай байдаг \cite{hcs}. 


\subsection{Elastic Compute Service}
HCS-ийн Elastic Compute Service (ECS) нь AWS-ийн EC2-тэй дүйх бөгөөд аливаа сүлжээний subnet дээр орших виртуал машин юм \cite{aws}\cite{hcs}. 

\subsection{Virtual Private Cloud}
HCS дотор орших тусгаарлагдсан Data Center-ийн дотоод сүлжээ юм. VPC-г зөв зохион байгуулах нь цаашид хандалтыг хязгаарлах, бусад гадны сүлжээтэй харилцах гэх мэт боломжуудыг олгоно \cite{hcs}.

\subsection{Virtual Private Network}
VPC-ийн дотоод сүлжээ нь гадны дотоод сүлжээтэй холбогдохын тулд VPN холболт үүсгэх шаардлагатай байдаг \cite{hcs}.

\begin{figure}[H]
\centering
\includegraphics[width=0.8\textwidth]{../images/vpn.png}
\caption{VPN холболтын диаграм}
\end{figure}

\section{Langfuse}

Langfuse нь LLM дээр суурилсан системүүдийн ажиглалт, хяналт болон гүйцэтгэлийн шинжилгээнд зориулагдсан open-source платформ юм \cite{langfuse}. Agentic AI архитектурын хувьд олон шатлалт reasoning, tool call, decision making процессууд явагддаг тул эдгээрийг бүртгэн хадгалах шаардлага үүсдэг. Langfuse нь prompt, response, latency, token usage зэрэг мэдээллүүдийг цуглуулж, LLM-ийн үйл ажиллагааг бодит орчинд хянах боломжийг олгодог.

Unitel Group нь Langfuse-г LLM-д суурилсан системүүдийг Production ба Development орчинд ашиглах үед алдааг илрүүлэх, гүйцэтгэлийг оновчлох мөн LLM-ийн Prompt mangement хийх зорилгоор ашигладаг.

\section{Gemini}

Gemini нь Google компанийн хөгжүүлсэн Large Language Model юм \cite{gemini}. Тус модельд embeddings үүсгэх чадвар 2025 онд шинээр нэмэгдсэн.


Gemini embedding нь текстийг семантик хайлтад ашиглах боломжтой 3072 хэмжээст вектор болгон хувиргадаг . Agentic AI архитектурын хүрээнд embeddings-ийг ашигласнаар систем нь хадгалагдсан мэдлэгтэй уялдуулан шийдвэр гаргах боломж бүрддэг.

\section{AWS Cohere}

Amazon Web Service-ийн санал болгодог service-үүдйин нэг болох Cohere нь Large Language Model болон embeddings үйлчилгээ үзүүлэгч  юм\cite{cohere}.

Unitel Group саяхныг хүртэл Cohere Multilingual v3 embedding моделийг ашигладаг байсан бөгөөд 2026 оны 1 сард хийсэн туршилт дээр үндэслэн Cohere-ийг fallback модель, харин gemini-г үндсэн embedding модель болгон шилжүүлсэн.

\section{Github}

Github нь Git reposistory host service үзүүлэгч юм \cite{github}. Мөн үүнтэй уялдуулан Job pipeline (Github Actions), discoverability болон эх код дээр хамтран ажиллах гэх мэт нэмэлт үйлчилгээ үзүүлдэг.

\subsection{Github self-hosted runner}

Github self-hosted runner нь Github Actions workflow-г тухайн хэрэглэгчийн өөрийн дэд бүтцэд байршуулсан server дээр гүйцэтгэх боломж олгодог механизм юм \cite{github}. Cloud-based runner-ээс ялгаатай нь self-hosted runner нь дотоод сүлжээ, Data Center болон тусгаарлагдсан орчинд байрших системүүдтэй шууд харилцах боломжтой.

Дата Сайнсын хэлтсийн хувьд Github ашиглан CI/CD болон эх кодоо удирддаг учраас Github self-hosted runner ашиглан CI/CD процессыг дотоод орчинд гүйцэтгэдэг.


\subsection{Github Actions}

Github Actions нь программ хангамжийн хөгжүүлэлт, тест, нэвтрүүлэлтийн (CI/CD) процессыг автоматжуулах зориулалттай workflow orchestration систем юм \cite{github}. Workflow нь event-д суурилан ажилладаг бөгөөд кодын өөрчлөлт, pull request, эсвэл гар ажиллагаатай trigger-ээр эхлэн олон шатлалт процессыг гүйцэтгэх боломжтой.

\section{Google Cloud Platform}
Google Cloud Platform (GCP) нь Google компанийн хөгжүүлсэн cloud computing үйлчилгээний цогц платформ бөгөөд compute, storage, networking, data analytics болон хиймэл оюун ухаанд суурилсан олон төрлийн үйлчилгээг санал болгодог. GCP-гийн өндөр найдвартай байдал, аюулгүй байдлын механизм, автоматжуулалтын боломжуудтай гэдгээрээ Дата Сайнсын хэлтсийн хэрэглээнд тохиромжтой тул өөрсдийн систем болон pipeline-ууддаа ашигладаг.

\subsection{Cloud Build}
Cloud Build нь Google Cloud Platform-ийн Continuous Integration / Continuous Deployment (CI/CD) үйлчилгээнүүдийн нэг бөгөөд автоматаар build, test болон deploy хийх боломж олгодог \cite{gcp}. Cloud Build нь event-д суурилан ажилладаг бөгөөд GitHub зэрэг source control системүүдтэй шууд уялдан ажиллах чадвартай гэдгээрээ давуу талтай.

Энэхүү сервисийг ашигласнаар аливаа server дээр тусгай build орчин урьдчилан бэлдэх шаардлагагүй бөгөөд build процесс бүр тусгаарлагдсан, аюулгүй орчинд гүйцэтгэгддэг. Энэ нь deployment процессийн тогтвортой байдал, давтагдах чанар болон аюулгүй байдлыг хангахад чухал үүрэгтэй.

\subsection{Workload Identity Federation}
Workload Identity Federation (WIF) нь гадны орчинд ажиллаж буй workload-уудаас (GitHub Actions) ирж буй хүсэлтүүдийг баталгаажуулахад ашиглагддаг Google Cloud Platform-ийн authentication механизм юм.

WIF ашигласнаар workload нь өөрийн identity (OIDC token)-г ашиглан Google Cloud-ийн service account-т түр хугацааны authenthication ашиглан event trigger хийх зарчмаар ажилладаг.\. Ингэснээр static key хадгалах шаардлагагүй болж байгаа учир Service Account Key алдагдах эрсдэлийг бууруулдаг.

Цаашлаад хандалтыг илүү нарийн, policy-д суурилан удирдах боломжтой болно

Энэхүү механизм нь enterprise орчинд CI/CD процессыг Google Cloud-той аюулгүй, автомат байдлаар холбох хамгийн зөв шийдэлд тооцогддог.

\subsection{Artifact Registry}
Artifact registry нь Google Cloud platform дээр build хадгалах storage систем юм. Тус service-ийг Cloud Build дээр хийсэн build-үүдээ Production server-т дамжуулахын тулд ашиглана.

\section{Qdrant}
Qdrant нь өндөр гүйцэтгэлтэй, Open-Source вектор өгөгдлийн сан бөгөөд их хэмжээний embedding векторууд дээр ойролцоо хайлт (Approximate Nearest Neighbor) гүйцэтгэхэд зориулагдсан систем юм \cite{qdrant}. Тус систем нь cosine similarity, dot product болон Euclidean distance зэрэг олон төрлийн similarity хэмжүүрийг дэмждэг бөгөөд Large Language Model (LLM)-д суурилсан Retrieval-Augmented Generation (RAG), semantic search болон recommendation системүүдэд өргөн ашиглагддаг.

Qdrant нь metadata filtering, tenant-based separation болон access control зэрэг enterprise түвшний боломжуудыг дэмждэгээрээ давуу талтай бөгөөд self-hosted орчинд ашиглах боломжтой. 