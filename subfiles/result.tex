Дадлагын ажлын хүрээнд Unitel Group компанийн Дата сайнсын хэлтсийн дотоод системүүдэд хиймэл оюун ухаанд суурилсан шийдлүүдийг судалж, бодит орчинд хэрэгжүүлсэн үр дүнгүүд дээр дүгнэлт хийвэл.

Agentic AI архитектур болон Model Context Protocol (MCP)-д суурилсан шийдлүүдийг нэвтрүүлснээр AI туслахуудын гүйцэтгэх боломжит үйлдлүүд өргөжсөн. Үүнд
    \begin{itemize}
        \item Qdrant вектор өгөгдлийн сан болон gemini embedding модель ашигласнаар Retrieval MCP-ний хурд 4 дахин сайжирсан.
        \item ERP MCP-ний дундаж гүйцэтгэлийн хурд 10 хувь багасаж, гуравдагч системийн доголдлыг зохицуулах логик сайжруулсан. 
    \end{itemize}

TV Content дата оруулалтын ажлыг автоматжуулснаар хоёр ажилтны 16 цагийн ажлыг нийт 1 хүн/цаг болгон бууруулж байгаа.

Үүнээс гадна Huawei Cloud Stack орчинд development болон production server-үүдийн архитектурыг зохион байгуулснаар дотоод орчны тусгаарлалт, аюулгүй байдал болон deployment процессыг сайжруулсан нь:
    \begin{itemize}
        \item Server уналтыг багасгах.
        \item Development server-ийн ачааллыг багасгах.
        \item Production service-үүдийг тусгаарлагдсан орчинд удирдсанаар бусад service-ээс хамаарсан доголдол гарахгүй байх гэх мэт давуу талуудыг олгож байгаа.
    \end{itemize}

